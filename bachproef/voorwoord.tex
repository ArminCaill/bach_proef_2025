%%=============================================================================
%% Voorwoord
%%=============================================================================

\chapter*{\IfLanguageName{dutch}{Woord vooraf}{Preface}}%
\label{ch:voorwoord}

%% TODO:
%% Het voorwoord is het enige deel van de bachelorproef waar je vanuit je
%% eigen standpunt (``ik-vorm'') mag schrijven. Je kan hier bv. motiveren
%% waarom jij het onderwerp wil bespreken.
%% Vergeet ook niet te bedanken wie je geholpen/gesteund/... heeft

De keuze voor het onderwerp van deze bachelorproef is ontstaan uit mijn interesse voor zowel cybersecurity als automatisering.
Door het vak Infrastructure Automation, waar we in aanraking kwamen met Ansible, kreeg ik nog meer bewondering voor dit deel van de IT wereld.
Ook het DevOps-project was een leerrijke ervaring waarin we onze verworven kennis konden gebruiken voor het opzetten van onze eigen infrastructuur.

In een wereld waarin cloudinfrastructuren steeds complexer worden en cyberdreigingen toenemen, 
vind ik het belangrijk om beveiligingsrisico's te minimaliseren door een consistent configuratieproces toe te passen.
Microsoft Azure wordt door veel organisaties gebruikt om virtuele machines te hosten, maar het handmatig beheren en beveiligen van deze omgevingen brengt de nodige uitdagingen met zich mee.
Met dit onderzoek wil ik aantonen hoe configuration management met Ansible kan bijdragen aan een efficiëntere en veiligere cloudomgeving.

Dit onderzoek was niet mogelijk geweest zonder de steun en begeleiding van enkele belangrijke personen.
Eerst en vooral wil ik mijn promotor en co-promotor bedanken voor hun waardevolle feedback en ondersteuning gedurende dit traject.
Daarnaast wil ik mijn medestudenten en docenten van de opleiding Toegepaste Informatica aan HoGent bedanken voor de inspirerende gesprekken die hebben bijgedragen aan mijn leerproces.
Tot slot ben ik ook mijn familie en vrienden dankbaar voor hun onvoorwaardelijke steun en motivatie.

Ik hoop dat deze bachelorproef bijdraagt aan een beter begrip van de rol van automatisering in cybersecurity en 
dat mijn bevindingen relevant zullen zijn voor professionals die werken met cloudinfrastructuren.
