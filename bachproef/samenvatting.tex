%%=============================================================================
%% Samenvatting
%%=============================================================================

% TODO: De "abstract" of samenvatting is een kernachtige (~ 1 blz. voor een
% thesis) synthese van het document.
%
% Een goede abstract biedt een kernachtig antwoord op volgende vragen:
%
% 1. Waarover gaat de bachelorproef?
% 2. Waarom heb je er over geschreven?
% 3. Hoe heb je het onderzoek uitgevoerd?
% 4. Wat waren de resultaten? Wat blijkt uit je onderzoek?
% 5. Wat betekenen je resultaten? Wat is de relevantie voor het werkveld?
%
% Daarom bestaat een abstract uit volgende componenten:
%
% - inleiding + kaderen thema
% - probleemstelling
% - (centrale) onderzoeksvraag
% - onderzoeksdoelstelling
% - methodologie
% - resultaten (beperk tot de belangrijkste, relevant voor de onderzoeksvraag)
% - conclusies, aanbevelingen, beperkingen
%
% LET OP! Een samenvatting is GEEN voorwoord!

%%---------- Nederlandse samenvatting -----------------------------------------
%
% TODO: Als je je bachelorproef in het Engels schrijft, moet je eerst een
% Nederlandse samenvatting invoegen. Haal daarvoor onderstaande code uit
% commentaar.
% Wie zijn bachelorproef in het Nederlands schrijft, kan dit negeren, de inhoud
% wordt niet in het document ingevoegd.

\IfLanguageName{english}{%
\selectlanguage{dutch}
\chapter*{Samenvatting}
\lipsum[1-4]
\selectlanguage{english}
}{}

%%---------- Samenvatting -----------------------------------------------------
% De samenvatting in de hoofdtaal van het document

\chapter*{\IfLanguageName{dutch}{Samenvatting}{Abstract}}

De snelle groei van cloudplatformen zoals Microsoft Azure biedt organisaties ongekende flexibiliteit en schaalbaarheid.
Toch brengt het handmatig beheren en beveiligen van Azure virtuele machines aanzienlijke uitdagingen met zich mee.
Inconsistenties in het configuratieproces, menselijke fouten en tijdrovend beheer kunnen leiden tot verhoogde operationele kosten en beveiligingsrisico's.
Deze bachelorproef onderzoekt hoe configuration management met Ansible kan bijdragen aan een efficiënter en veiliger beheer van Azure VM's.

Het onderzoek vertrekt vanuit de probleemstelling dat veel organisaties moeite hebben om hun cloudinfrastructuur op een consistente en veilige manier te configureren.
De centrale onderzoeksvraag luidt: Hoe kan Ansible bijdragen aan de automatisering en beveiliging van Azure virtuele machines?
De doelstelling van dit onderzoek is om een oplossing te ontwikkelen die het beheer van Azure VM's optimaliseert en standaardiseert, met een specifieke focus op cybersecurity.

Om deze vraag te beantwoorden, werd een experimentele onderzoeksmethode toegepast.
Eerst werd de traditionele, handmatige manier van VM-configuratie en -beveiliging geanalyseerd en geëvalueerd.
Vervolgens werd een reeks Ansible playbooks ontwikkeld om de implementatie van VM's te automatiseren en te standaardiseren.
Deze geautomatiseerde aanpak werd getest en vergeleken met de handmatige werkwijze op basis van efficiëntie, consistentie en beveiliging.

Uit de resultaten blijkt dat Ansible een aanzienlijke tijdsbesparing oplevert bij de configuratie van Azure VM's en de kans op menselijke fouten drastisch vermindert.
Daarnaast zorgt de implementatie van beveiligingsrichtlijnen in Ansible playbooks voor een verhoogde bescherming tegen veelvoorkomende cloudgerelateerde beveiligingsrisico's.
Door configuratiebeheer te automatiseren, kunnen bedrijven een consistente en conforme cloudomgeving realiseren.

De bevindingen van dit onderzoek zijn relevant voor systeem- en netwerkbeheerders die cloudomgevingen beheren en optimaliseren.
Hoewel de studie zich toespitst op Azure, kunnen de methoden ook toegepast worden op andere cloudplatformen.
Toekomstig onderzoek kan zich richten op de integratie van Ansible met andere securitytools en de impact van automatisering op grootschalige cloudinfrastructuren.
