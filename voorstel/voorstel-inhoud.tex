%---------- Inleiding ---------------------------------------------------------

% TODO: Is dit voorstel gebaseerd op een paper van Research Methods die je
% vorig jaar hebt ingediend? Heb je daarbij eventueel samengewerkt met een
% andere student?
% Zo ja, haal dan de tekst hieronder uit commentaar en pas aan.

%\paragraph{Opmerking}

% Dit voorstel is gebaseerd op het onderzoeksvoorstel dat werd geschreven in het
% kader van het vak Research Methods dat ik (vorig/dit) academiejaar heb
% uitgewerkt (met medesturent VOORNAAM NAAM als mede-auteur).
% 

\section{Inleiding}%
\label{sec:inleiding}

Door de snelle groei van cloudplatformen hebben bedrijven toegang tot ongekende flexibiliteit en schaalbaarheid ten opzichte van het beheren van een eigen fysieke infrastructuur.
Microsoft Azure is een van de meest gebruikte cloudplatformen ter wereld, dat bedrijven de mogelijkheid biedt om virtuele machines op te zetten en te beheren.
Hoewel dit platform veel voordelen biedt, brengt het ook nieuwe uitdagingen met zich mee op het gebied van cybersecurity en de complexiteit van de infrastructuur.
Voor bedrijven die dagelijks afhankelijk zijn van hun cloudinfrastructuur kan inefficiënt beheer van deze virtuele machines leiden tot hogere operationele kosten, inconsistenties in configuratie en kwetsbaarheden in de beveiliging.
Een concreet probleem ontstaat bij snelgroeiende organisaties met een steeds veranderende Azure-omgeving.
De virtuele machines worden handmatig beheerd, waardoor configuraties verschillen en belangrijke beveiligingsinstellingen niet uniform worden toegepast.
Hierdoor kunnen er zich beveiligingsincidenten voordoen en kunnen de bedrijven slachtoffer worden van cyberaanvallen.

De probleemstelling richt zich op twee cruciale punten:

\begin{itemize}
  \item Automatisatie: Kunnen we met ansible playbooks het uitrollen van virtuele machines efficiënter maken op vlak van tijdsduur?
  \item Consistentie: Kan het uitrollen van virtuele machines worden gestandaardiseerd om\linebreak menselijke fouten te minimaliseren op vlak van beveiliging?
\end{itemize}

Met dit onderzoek wordt aangetoond hoe configuration management, met tools zoals Ansible, kan bijdragen aan een veilige en\linebreak geautomatiseerde cloudomgeving.
Het doel is om een schaalbare oplossing te ontwikkelen die zowel het beheer vereenvoudigt als de beveiliging van Azure virtuele machines verbetert.

%---------- Stand van zaken ---------------------------------------------------

\section{Literatuurstudie}%
\label{sec:literatuurstudie}

\subsection{Config Management}%

Voor veel bedrijven is het belangrijk om de concurrentie voor te blijven, waardoor deze genoodzaakt zijn om vlug op nieuwe trends in te gaan.
Dit kan ervoor zorgen dat ze snel groeien en nood hebben aan een meer uitgebreide infrastructuur dan origineel voorzien was.
Eerdere studies tonen aan dat \emph{Infrastructure as Code} een kritieke rol speelt in het automatiseren van configuratie en management van cloudomgevingen\linebreak~\autocite{Kalliomaa2024}.
Hoewel er meerdere tools bestaan voor \emph{config management} wordt er hier gekozen voor Ansible.
Omdat Ansible agentless is, hoeft er geen extra software geïnstalleerd te worden, waardoor het compatibel is met een breed scala aan omgevingen~\autocite{Elradi2023}.
Door het gebruik van Ansible playbooks kunnen deze bedrijven een infrastructuur opbouwen die zowel veilig als schaalbaar is en makkelijk aanpasbaar aan de noden van het bedrijf.

\subsection{Security}%

Deze schaalbaarheid brengt op zich nieuwe problemen met zich mee op het gebied van beveiliging.
Volgens~\textcite{Ots2021} onstaan de meest voorkomende securitybedreigingen door\linebreak verkeerde configuratie van cloudservices, en niet door aanvallen van buitenaf.
Het beveiligen van cloudomgevingen vereist een proactieve aanpak, waaronder het toepassen van beveiligingsrichtlijnen en het bewaken van compliance.
Voor groeiende bedrijven kan ook het onboarding en offboarding proces moeilijkheden met zich meebrengen op gebied van toegangsrechten.
Door de implementatie van config management, kunnen beveiligingsinstellingen zoals netwerkbeperkingen, toegangspolicies en encryptie worden\linebreak opgenomen in Ansible playbooks.
Dit zorgt voor een consistente naleving van de beveiligingsstandaarden.

% Voor literatuurverwijzingen zijn er twee belangrijke commando's:
% \autocite{KEY} => (Auteur, jaartal) Gebruik dit als de naam van de auteur
%   geen onderdeel is van de zin.
% \textcite{KEY} => Auteur (jaartal)  Gebruik dit als de auteursnaam wel een
%   functie heeft in de zin (bv. ``Uit onderzoek door Doll & Hill (1954) bleek
%   ...'')

%---------- Methodologie ------------------------------------------------------
\section{Methodologie}%
\label{sec:methodologie}

\subsection{Probleemanalyse}%

De eerste stap is het vaststellen van de exacte uitdagingen bij het beheer van een Azure infrastructuur en het in kaart brengen van relevante praktijken, technieken en tools.
Dit omvat een grondige literatuurstudie van wetenschappelijke artikelen en technische documentatie over Azure, Ansible en cybersecurity principes.
Het doel van dit literatuuronderzoek is om een diepgaand inzicht te verkrijgen in bestaande beveiligings- en beheerstrategieën voor Azure-omgevingen, evenals de rol die configuration management tools kunnen spelen.
Deze fase resulteert in een heldere probleemstelling en een verzameling van best practices die als uitgangspunt dienen voor de rest van het onderzoek.

\begin{itemize}
  \item Tijdsindicatie: 3 weken
\end{itemize}

\subsection{Traditionele opbouw}%

In deze fase wordt een experimentele omgeving handmatig opgebouwd.
Hier is het belangrijk de tijd te registreren zodat we deze later kunnen vergelijken met het opbouwen aan de hand van playbooks.
Er wordt ook een gedetailleerd overzicht opgesteld van de huidige VM-\linebreak configuraties en beveiligingsinstellingen zodat we deze later kunnen vergelijken.
Eventuele tekortkomingen op vlak van beveiliging worden geïdentificeerd aan de hand van de probleemanalyse.

\begin{itemize}
  \item Tijdsindicatie: 3 weken
\end{itemize}

\subsection{Ontwerp van playbooks}%

Op basis van de inzichten uit de voorgaande fasen wordt een oplossing ontwikkeld.
Deze bestaat uit Ansible playbooks die virtuele machines configureren en beveiligen volgens gestandaardiseerde processen.
De ontwikkeling gebeurt iteratief, waarbij telkens kleine componenten worden getest en verfijnd.

\begin{itemize}
  \item Tijdsindicatie: 5 weken
\end{itemize}

\subsection{Opbouw met Ansible}%

De ontwikkelde oplossing zal gebruikt worden om de experimentele omgeving opnieuw op te bouwen.
In deze fase wordt de veiligheid van de virtuele machines opnieuw beoordeeld. 
Er wordt ook opnieuw een gedetailleerd overzicht opgesteld om de verbeteringen ten opzichte van de oorspronkelijke situatie te bekijken.

\begin{itemize}
  \item Tijdsindicatie: 3 weken
\end{itemize}

\subsection{Evaluatie en documentatie}%

In de laatste fase worden de behaalde resultaten geëvalueerd en gedocumenteerd.
Er wordt een vergelijking gemaakt tussen de tijdsduur van het traditioneel handmatig opbouwen en het gebruik van playbooks.
Ook worden de verschillen en verbeteringen in kaart gebracht op het gebied van beveiliging.
Het eindrapport bevat een uitgebreide beschrijving van het onderzoek, inclusief de probleemstelling, methodologie, resultaten en aanbevelingen.
Daarnaast wordt een presentatie voorbereid voor de stakeholders van het stagebedrijf en de onderwijsinstelling, waarin de waarde en impact van de oplossing wordt toegelicht.

\begin{itemize}
  \item Tijdsindicatie: 2 weken
\end{itemize}

%---------- Verwachte resultaten ----------------------------------------------
\section{Verwacht resultaat}%
\label{sec:verwachte_resultaten}

Dit onderzoek heeft als doel om een efficiënter, veiliger en consistenter beheer van cloudinfrastructuur te realiseren.
Door gebruik te maken van Ansible voor de automatisatie van configuraties en beveiligingsinstellingen kan het handmatige werk van systeem- en netwerkbeheerders aanzienlijk worden verminderd.
Op basis van de evaluatie wordt een conclusie getrokken over de meerwaarde van Ansible voor het beheren en beveiligen van Azure virtuele machines.
